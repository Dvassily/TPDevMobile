\documentclass{article}
\usepackage[utf8]{inputenc}
\usepackage{graphicx}
\title{TP1 - Développement mobile}
\author{Basil Dalié}
\date{23 Février 2020}

\begin{document}

\maketitle

\section{Question}
Parmi la suite Android Studio on trouve :
\begin{itemize}
\item L'IDE android studio
\item Le gestionnaire de péripheriques virtuels (AVD Manager)
\item Le SDK manager qui permet d'installer et configurer des modules du SDK android
\end{itemize}

\section{Question 2}
Parmi les fichiers dans un projet vide Android Studio, on trouve nottamment
\begin{itemize}
\item Le fichier AndroidManifest.xml
  Il définit un ensemble de métadonnées associée à une application android telle que le nom de l'application, l'icône à afficher, l'activité de départ, les permissions, etc.
\item Le fichier MainActivity.java
  Il s'agit d'une classe proposant entre autre la méthode onCreate qui permet de construire l'interface associée à une activité et de définir les réponses aux actions utilisateur sur ses composants.
\item Le répertoire res/drawable
  Il contient l'ensemble des images pouvant être utiliées par l'application en temps que composant graphique ou icone
\item Le fichier layout/main\textunderscore activity.xml
  Il définit une hierarchie de composant graphiques et est par défaut associé à l'activité principale de l'application.
\item Le fichier values/strings.xml
  Il contient un dictionnaire dont les valeurs sont les chaînes de caractères situées dans l'application. On peut en créer plusieurs associée à une langue différente afin d'assurer la traduction par le smartphone.
\item Le fichier app/build.gradle
  Il définit des options de compilation à destination du système gradle telle que les versions du SDK couvertes par l'appliacation, le numéro de version, les dépendances, etc.
\end{itemize}

\section{Question 3 à 9}
Code source est image sur le dépot GitHub :
https://github.com/Dvassily/TPDevMobile

\end{document}
